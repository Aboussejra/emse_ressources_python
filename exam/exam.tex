\documentclass[12pt,a4paper]{article}
\usepackage[utf8]{inputenc}
\usepackage{hyperref}

\setlength{\parskip}{0.8em}
\setlength{\parindent}{0pt}

\begin{document}

\begin{center}
\Large \textbf{Exercice de fin de cours – Introduction à Python}\\[0.5em]
\small Cursus Médecine–Ingénieur
\end{center}

Vous allez réaliser une analyse de données médicales en autonomie à l’aide de \texttt{Python}.  
Le jeu de données choisi peut provenir de n'importe quel source ouverte, voici des exemples:
\begin{itemize}
    \item \href{https://www.kaggle.com/datasets}{Kaggle}
    \item \href{https://archive.ics.uci.edu/datasets}{UCI Machine Learning Repository}
    \item \href{https://data.gouv.fr/}{Data.gouv.fr}
\end{itemize}
Choisissez un ensemble de données qui vous intéresse, quel que soit le domaine médical !  

\section*{Travail demandé}

Proposez une analyse scientifique libre en utilisant les outils et librairies Python que vous souhaitez, en exemple les plus classiques :
\begin{itemize}
    \item \texttt{pandas/polars} (pour le chargement et la manipulation des données : tables, tris, filtres). \\
          Documentation officielle Pandas : \href{https://pandas.pydata.org/docs/}{https://pandas.pydata.org/docs/} \\
          Polars : \href{https://www.pola.rs/}{https://www.pola.rs/}
    \item \texttt{numpy} (pour les calculs numériques, statistiques simples) \\
          Documentation officielle NumPy : \href{https://numpy.org/doc/}{https://numpy.org/doc/}
    \item \texttt{matplotlib/seaborn} (pour générer des graphiques) \\
          matplotlib : \href{https://matplotlib.org/stable/contents.html}{https://matplotlib.org/stable/contents.html} \\
          seaborn : \href{https://seaborn.pydata.org/}{https://seaborn.pydata.org/}
\end{itemize}

\newpage
\section*{Rendu attendu}

Le rendu sera un PDF contenant :
\begin{itemize}
    \item les résultats graphiques et statistiques,
    \item le code commenté vous ayant permis d'avoir ce résultat,
    \item une analyse/interprétation personnelle de vos observations des données choisies.
\end{itemize}
N'hésitez pas à explorer librement ! L'important est d'apprendre à utiliser Python dans un contexte médical, et à communiquer vos découvertes (et non d'écrire du code robuste d'ingénieur logiciel).

P.S.: Les plus motivés peuvent tester la création d’un dashboard interactif (voir librairie \texttt{streamlit} : \href{https://streamlit.io/}{https://streamlit.io/}), voire même l'héberger en ligne mais c’est optionnel et plus du travail d'ingénieur qu'autre chose.

\end{document}
